\documentclass[landscape]{article}
\usepackage{courier}
\renewcommand*\familydefault{\ttdefault} %% Only if the base font of the document is to be typewriter style
\usepackage[T1]{fontenc}
\usepackage[a4paper]{geometry}
\usepackage{wallpaper}
\usepackage{niceframe}
\usepackage{xcolor}
\usepackage{ulem}
\usepackage{graphicx}
\usepackage{geometry}
\geometry{tmargin=0.5cm,bmargin=.5cm,
lmargin=.5cm,rmargin=.5cm}
\usepackage{multicol}
\setlength{\columnseprule}{0.4pt}
\columnwidth=0.3\textwidth

\usepackage{lipsum}

\usepackage{soul}
\usepackage[doublespacing]{setspace}

\renewcommand{\baselinestretch}{1.0} 

\usepackage{authblk}
\usepackage{hyperref}

\newcommand*{\email}[1]{%
    \normalsize\href{mailto:#1}{#1}\par
    }

\begin{document}

%\TileWallPaper{4cm}{4cm}{background_grayscale.png}

\centering
\scalebox{2.8}{\color{green!30!black!60}
\begin{minipage}{.33\textwidth}
{\centering

\vspace{1.5mm}

\begin{minipage}{1\textwidth}
\centering
\includegraphics[height=.8cm]{../imagens/unihacker_1oChuHask_top.png}\hfill 
{\color{white}x} \hfill
\includegraphics[height=0.9cm]{../imagens/unihacker_1oDataCenterPilot_Interneith_logo.jpg}
\end{minipage}

\vspace{-6mm}

{\large \bf Clube Universidade Hacker}\\

\vspace{2mm}

\textcolor{green!5!black!95}{
\tiny {1o Data Center Pilot (Etapa 1), evento técnico, parte do Programa UniHacker.Club, juntando a comunidade interna e externa no PampaTec, realizado no dia 03 de junho de 2019 em Alegrete-RS, Brasil.}}

\vspace{3mm}

% TIPO: OUVINTE, ORGANIZADOR, PALESTRANTE, ...
\textcolor{red!30!black!90}
{Certificado de} \textcolor{black}{\large \textsc{TAGTIPO}}

\vspace{2mm}

\textcolor{blue!30!black!90}{
\footnotesize
% NOME DO INDIO VEIO
para \textbf{TAGNOME}}

\vspace{-1mm}

\textcolor{black}{
\tiny
% NUMERO DE HORAS
{(TAGHORAS hora(s) de atividades)}
}

\vspace{2mm}
}
\end{minipage}
}

\vspace{3mm}

\begin{minipage}{.9\textwidth}
\centering
{\color{blue!10!black}
\scalebox{1.6}{
\begin{tabular}{cccc}
% ASSINANTES DO CERTIFICADO
Diego Kreutz  & &  Deivid Rodrigues & \\
Coordenador do Programa & & Organizador do Evento & \\ 
\email{kreutz@unipampa.edu.br} & & \email{deivid@interneith.com.br} & \\ 
\end{tabular}
}}
\end{minipage}

\vfill

\begin{minipage}{.9\textwidth}
\centering
\scalebox{1.8}{
%\hfill 
\includegraphics[height=1.5cm]{./qr_code_url.png}\hspace{7mm}
\includegraphics[height=1.5cm]{./qr_code_url_sha256.png}\hspace{7mm}
\includegraphics[height=1.5cm]{./qr_code_url_gpg.png}\hspace{7mm}
\includegraphics[height=1.6cm]{../imagens/unihacker_1oChuHask_bottom.png}\hspace{7mm}
\includegraphics[height=1.5cm]{./qr_code_gpg_key.png}\hspace{7mm}
\includegraphics[height=1.5cm]{./qr_code_hmac_qr_codes.png}
}
\end{minipage}

\vfill

\begin{minipage}{.9\textwidth}
\centering
{\color{black}
\scalebox{1.4}{
E-mail: \email{info@unihacker.club} \hspace{4mm} Web: \url{https://unihacker.club}

{\tiny ** O \textbf{Programa UniHacker.Club} está registrado sob o número \textbf{01.023.19} no SIPPEE da UNIPAMPA. **}
}
}
\end{minipage}

\vfill
%\vspace{2mm}

\end{document}
